%\documentclass[class=article, crop=false]{standalone}
\documentclass{article}
\usepackage{amssymb,amsfonts,amsthm,amsmath}
\usepackage{enumitem}
\usepackage{hyperref,xcolor}
\hypersetup{
    colorlinks,
    urlcolor={blue}  %black!50!blue
}
\renewcommand{\theequation}{\theenumi}
\providecommand{\cbrak}[1]{\ensuremath{\left\{#1\right\}}}
\providecommand{\brak}[1]{\ensuremath{\left(#1\right)}}
\newcommand{\solution}{\noindent \textbf{Solution: }}
%\newcommand{\varsol}{\noindent \textbf{Aliter: }}
\newcommand*{\permcomb}[4][0mu]{{{}^{#3}\mkern#1#2_{#4}}}
%\newcommand*{\perm}[1][-3mu]{\permcomb[#1]{P}}
\newcommand*{\comb}[1][-1mu]{\permcomb[#1]{C}}
\setlist[enumerate]{font=\small\bfseries}
\renewcommand\thefootnote{\textcolor{black}{\arabic{footnote}}}
\providecommand{\pr}[1]{\ensuremath{\Pr\left(#1\right)}}

\begin{document}

\title{PROBABILITY}
\author{\Large Rupa Sai Sreshta Vallabhaneni}
\date{}

\maketitle
\begin{enumerate}[label=13.\arabic{enumi}.\arabic{enumii}]%,ref=\thesection.\theenumi.\theenumi]
\numberwithin{equation}{enumi}
\setcounter{enumi}{1}
\setcounter{enumii}{9}

\item \footnote{Read question numbers as (CHAPTER NUMBER).(EXERCISE NUMBER).(QUESTION NUMBER)} If A and B are two events such that $\pr{A} = \frac{1}{4}, \pr{B} = \frac{1}{2}$ and $\pr{A B} = \frac{1}{8}$. find $\pr{\text{not A and not B}}$.

\solution
\item 
\begin{align}
A^{\prime}B^{\prime} &=  \brak{A+B}^{\prime}
\\
\implies \pr{A^{\prime}B^{\prime}} &=  \pr{\brak{A+B}^{\prime}} 
\\
&= 1 - \pr{A+B} 
\label{eq:1}
\end{align} 
\item 
\begin{align}
 A+B &= A\brak{B+B^{\prime}} + B
\\
&= B \brak{A +1} + A B^{\prime}
\\
&=B + A B^{\prime}
\\
\implies \pr{A+B} &= \pr{B + A B^{\prime} }
\\
&=\pr{B}+\pr{ A B^{\prime} } 
\\
&B \brak{ A B^{\prime} } = 0
\label{eq:2}
\end{align}
\item 
\begin{align}
A = A \brak{B+B^{\prime}} =  AB + AB^{\prime}
\label{eq:3}
\end{align}
and 
\begin{align}
\brak{ AB}\brak{  AB^{\prime}} = 0, 
\\
 BB^{\prime} = 0
\label{eq:4}
\end{align}
Hence, $AB$ and $AB^{\prime}$ are mutually exclusive and 
%
\begin{align}
\pr{A} = \pr{AB} + \pr{AB^{\prime}}
\\
\implies 
\pr{AB^{\prime}} =  \pr{A} - \pr{AB}
\label{eq:5}
\end{align}
\item Substituting \eqref{eq:5} in \eqref{eq:2}, 
\begin{align}
\pr{A+B} &= \pr{A} + \pr{B} - \pr{AB} 
\label{eq:6}
\end{align}
\item Substituting \eqref{eq:6} in \eqref{eq:1}
\begin{align}
\pr{A^{\prime}B^{\prime}} &=  1 - \cbrak{\pr{A} + \pr{B} - \pr{AB} }
\\
&= 1 - \brak{\frac{1}{4} + \frac{1}{2} - \frac{1}{8}}
\\
&= \frac{3}{8}
\label{eq:7}
\end{align}

































\end{enumerate}
\end{document}